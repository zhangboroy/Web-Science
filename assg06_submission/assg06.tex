\documentclass{article}
	\usepackage{geometry}
	\geometry{left=1.0cm,right=1.0cm}
	\usepackage{graphicx}
	\usepackage{indentfirst}
	\usepackage[hyphens]{url}
	\usepackage{listings}
	\usepackage{color}
	\definecolor{gray}{rgb}{0.3,0.3,0.3}
	\usepackage{caption}
	\DeclareCaptionFont{white}{\color{white}}
	\DeclareCaptionFormat{listing}{\colorbox{gray}{\parbox{\textwidth}{#1#2#3}}}
	\captionsetup[lstlisting]{format=listing,labelfont={bf,sf,white},textfont={bf,sf,white}}
	\usepackage{xcolor}
	\usepackage{booktabs}
	\captionsetup[table]{labelfont=bf}
	\usepackage[colorlinks,linkcolor=red]{hyperref}

	\begin{document}
		\begin{center}\textbf{Bo Zhang\\01063214}
		\end{center}
		\section{D3 graphing}
		Use D3 to visualize your Twitter followers. Use my twitter account if you do not have $>$= 50 followers.\\

		\noindent\textbf{Algorithm: }\\
		\indent1. Use the script from \url{http://stackoverflow.com/questions/31000178/how-to-get-large-list-of-
followers-tweepy} to download the information of all followers, and save them to \href{https://github.com/zhangboroy/cs532-s17/blob/master/assg06_submission/twitterFollowers.csv}{``twitterFollowers.csv''}.\\
		\indent2. Open \href{https://github.com/zhangboroy/cs532-s17/blob/master/assg06_submission/twitterFollowers.csv}{``twitterFollowers.csv''}, read all kinds of the information to different lists.\\
		\indent3. Use the package ``gender-detector''(\url{https://pypi.python.org/pypi/gender-detector/0.1.0}) to guess the gender and save it to another list.\\
		\indent4. Select 100 from the followers whose gender could be guessed, and mark them. Mark the main account as well.\\
		\indent5. Save all the followers as nodes to \href{https://github.com/zhangboroy/cs532-s17/blob/master/assg06_submission/graph.csv}{``graph.csv''}.\\
		\indent6. Use the package ``tweepy''(\url{http://docs.tweepy.org/en/latest/api.html}) to get the friendship within the 100 followers. If anyone of the 2 followers followed the other, save it as an edge to \href{https://github.com/zhangboroy/cs532-s17/blob/master/assg06_submission/graph.csv}{``graph.csv''}.\\
		\indent7. Open \href{https://github.com/zhangboroy/cs532-s17/blob/master/assg06_submission/graph.csv}{``graph.csv''}, read all the information and add the edges between the main account and the followers.\\
		\indent8. Use the package ``networkx''(\url{https://networkx.readthedocs.io/en/stable/}) to save the graph to \href{https://github.com/zhangboroy/cs532-s17/blob/master/assg06_submission/graph.json}{``graph.json''}.\\
		\indent9. Use the D3 to visualize the graph.\\\\
		\textbf{References:\\}
		\indent1. \url{https://bl.ocks.org/puzzler10/4efcb280a23c2f9b824879771ae41592}\\
		\indent2. \url{http://www.puzzlr.org/force-directed-graph-using-node-and-link-attributes/}\\
		\indent3. \url{http://stackoverflow.com/questions/18164230/add-text-label-to-d3-node-in-force-directed-graph-and-resize-on-hover}\\\\
		\textbf{Source code:}
		\lstinputlisting[language=python, breakatwhitespace=false), label=downloadTwitterFollowers.py, caption=The content of downloadTwitterFollowers.py]{downloadTwitterFollowers.py}
		\lstinputlisting[language=python, breakatwhitespace=false), label=FollowersCheck.py, caption=The content of FollowersCheck.py]{FollowersCheck.py}
		\lstinputlisting[language=python, breakatwhitespace=false), label=graphWrite-json.py, caption=The content of graphWrite-json.py]{graphWrite-json.py}
		\lstinputlisting[language=html, breakatwhitespace=false), label=Q1.html, caption=The content of Q1.html]{Q1.html}

		\noindent\\\textbf{Results: }\url{https://cdn.rawgit.com/zhangboroy/cs532-s17/b56ae831/assg06_submission/Q1.html}

		\section{Gender homophily in your Twitter graph}
		\indent Take the Twitter graph you generated in question \#1 and test for male-female homophily. For the purposes of this question you can consider the graph as undirected. Use the twitter name and programatically determine if the user is male or female.\\
		\indent Create a table of Twitter users and their likely gender. List any accounts that can't be determined and remove them from the graph. Does your Twitter graph exhibit gender homophily?\\

		\textbf{Algorithm:}\\
		\indent1. Open \href{https://github.com/zhangboroy/cs532-s17/blob/master/assg06_submission/graph.csv}{``graph.csv''}, read all the information and save it to an output list. Sort the output list with gender and save it to \href{https://github.com/zhangboroy/cs532-s17/blob/master/assg06_submission/genderTable.txt}{``genderTable.txt''}.\\
		\indent2. Select the 101 marked account from the output list and save them to another output list.\\
		\indent3. Compute the total edge number and account number of different genders.\\
		\indent4. Save the new output list to \href{https://github.com/zhangboroy/cs532-s17/blob/master/assg06_submission/genderTable100.txt}{``genderTable100.txt''}.\\
		\indent5. Compute the number of edges between different genders within the 101 accounts.\\
		\indent6. Print the account number of different genders, Randomly assigned cross-gender edge fraction, total edge number, cross-gender edge number and Actual cross-gender edge fraction.\\
		\indent7. Use the result from step 6 in R to run Exact Binomial Test.\\

		\noindent\textbf{Source code:}
		\lstinputlisting[language=python, breakatwhitespace=false), label=homophilyTest.py, caption=The content of homophilyTest.py]{homophilyTest.py}
		\lstinputlisting[language=R, breakatwhitespace=false), label=Q2.R, caption=The content of Q2.R]{Q2.R}

		\noindent\\\textbf{Results: }
		\lstinputlisting[language=R, breakatwhitespace=false), label=genderTable.txt, caption=The content of genderTable.txt]{genderTable.txt}
		\lstinputlisting[language=R, breakatwhitespace=false), label=genderTable100.txt, caption=The content of genderTable100.txt]{genderTable100.txt}
		\begin{table}[!htb]
			\centering
			\caption{\textbf{Gender Homophily Test Result}}
			\begin{tabular}{cccccc}
				\toprule
				\textbf{Male} & \textbf{Female} & \textbf{Total edges} & \textbf{Cross-gender edges} & \textbf{p-value} & \textbf{95 percent confidence interval}\\
				64 & 37 & 256 & 108 & 0.1882 & 0.3606438 - 0.4849366\\
				\bottomrule
			\end{tabular}
		\end{table}

		According to these figures, Randomly assigned cross-gender edge fraction is 0.4642682, Actual cross-gender edge fraction is 0.421875 and just 9\% off. The Exact Binomial Test p-value is 0.1882, which means the possibility of 81.2\% that they are not equal. So the gender homophily exists but slightly.\\
		\section{Using D3, create a graph of the Karate club before and after the split.}
		\indent Have the transition from before/after the split occur on a mouse click. This is a toggle, so the graph will go back and forth beween connected and disconnected.\\

		\textbf{Algorithm:}\\
		\indent1. Copy the last 34 lines of the matrix in \href{https://github.com/zhangboroy/cs532-s17/blob/master/assg06_submission/zachary.dat}{``zachary.dat''} and save them into \href{https://github.com/zhangboroy/cs532-s17/blob/master/assg06_submission/karate.txt}{``karate.txt''}.\\
		\indent2. Use the package ``igraph''(\url{http://igraph.org/r/}) in R to open \href{https://github.com/zhangboroy/cs532-s17/blob/master/assg06_submission/karate.txt}{``karate.txt''}.\\
		\indent3. Use the edge betweenness community detection algorithm.\\
		\indent4. Cut the merge tree to get 2 communities.\\
		\indent5. Plot the 2 communities.\\
		\indent6. Save the graph in ``edgelist'' format to \href{https://github.com/zhangboroy/cs532-s17/blob/master/assg06_submission/karateGraph.txt}{``karateGraph.txt''} and add the group index of the nodes to the file.\\
		\indent7. Open \href{https://github.com/zhangboroy/cs532-s17/blob/master/assg06_submission/karateGraph.txt}{``karateGraph.txt''}, for every edge check if the nodes are from different groups and add mark to the edge.\\
		\indent8. Use the package ``networkx''(\url{https://networkx.readthedocs.io/en/stable/}) to save the graph to \href{https://github.com/zhangboroy/cs532-s17/blob/master/assg06_submission/karate.json}{``karate.json''}.\\
		\indent9. Use the D3 to visualize the graph.\\\\
		\textbf{Source code:}
		\lstinputlisting[language=R, breakatwhitespace=false), label=karate.R, caption=The content of karate.R]{karate.R}
		\lstinputlisting[language=python, breakatwhitespace=false), label=graphWrite-karate.py, caption=The content of graphWrite-karate.py]{graphWrite-karate.py}
		\lstinputlisting[language=html, breakatwhitespace=false), label=karate.html, caption=The content of karate.html]{karate.html}

		\noindent\\\textbf{Results: }\url{https://cdn.rawgit.com/zhangboroy/cs532-s17/01870cb6/assg06_submission/karate.html}
	\end{document}